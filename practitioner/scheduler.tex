\documentclass[pdftex]{beamer}
\mode<presentation>
{
  \usetheme{default}
  \useoutertheme{infolines}
}

\usepackage[english]{babel}
\usepackage[latin1]{inputenc}
\usepackage{times}
\usepackage[T1]{fontenc}
\usepackage{fancyvrb}
\usepackage{listings}
\begin{document}
\lstset{language=C, escapeinside={(*@}{@*)}, numbers=left,
  basicstyle=\tiny, showspaces=false, showtabs=false, showstringspaces=false}

\title{A Look Inside FreeBSD with DTrace}
\subtitle{The Scheduler}
\author[shortname]{George V. Neville-Neil \and Robert N. M. Watson}

\begin{frame}
  \titlepage
\end{frame}
\begin{frame}[fragile]
  \frametitle{The Scheduler}
  \begin{itemize}
  \item Decides which thread gets to run
  \item The \emph{thread} is the scheduable entity
  \item Chooses a processor/core
  \item Can be overridden by \verb+cpuset+
  \end{itemize}
\end{frame}

\begin{frame}
  \frametitle{Process States}
  \begin{description}
  \item[NEW] Being created
  \item[RUNNABLE] Can run
  \item[SLEEPING] Awaiting some event
  \item[STOPPED] Debugging
  \item[ZOMBIE] Process of dying
  \end{description}
\end{frame}

\begin{frame}
  \frametitle{Scheduling Classes}
  \begin{description}
  \item[ITHD] interrupt thread
  \item[REALTIME] real-time user
  \item[KERN] kernel threads
  \item[TIMESHARE] normal user programs
  \item[IDLE] run when nothing else does
  \end{description}
\end{frame}

\begin{frame}[fragile]
  \frametitle{Scheduler Framework}
  \begin{itemize}
  \item Schedulers have kernel API
  \item \Verb|SCHED_4BSD| and \Verb|SCHED_ULE|
  \item High level scheduler picks the CPU via the runq
  \item Low level scheduler picks the thread to run
  \item \Verb|sched_pickcpu| selects the CPU
  \item \Verb|mi_switch| Entry to a forced context switch
  \item \Verb|sched_switch| scheduler API
  \end{itemize}
\end{frame}

\begin{frame}[fragile]
  \frametitle{Sched Provider}
  \begin{description}
  \item[on-cpu] Thread moves on core
  \item[off-cpu] Thread moves off core
  \item[remain-cpu] Thread remains on core
  \item[change-pri] Priority changed
  \item[fbt:kernel:cpu\_idle:entry] Thread went idle
  \end{description}
\end{frame}

\begin{frame}[fragile]
  \frametitle{Dummy Probes (Do Not Use)}
  \begin{itemize}
  \item Probes purely for D script compatibility
  \item These never fire
  \item \verb+cpucaps-sleep+
  \item \verb+cpucaps-wakeup+
  \item \verb+schedctl-nopreempt+
  \item \verb+schedctl-preempt+
  \item \verb+schedctl-yield+
  \end{itemize}
\end{frame}

\begin{frame}[fragile]
  \frametitle{Idle vs. Running}
  \begin{itemize}
  \item \verb+cpudists+
  \end{itemize}
\end{frame}

\begin{frame}[fragile]
  \frametitle{Who's sleeping?}
\begin{lstlisting}
dtrace -n 'sched:::sleep { @prog[execname] = count() }
dtrace: description 'sched:::sleep ' matched 1 probe
^C
cron                                      1
devd                                      1
pagezero                                  1
sendmail                                  1
sudo                                      1
nfsd                                          2
\end{lstlisting}
\end{frame}

\begin{frame}[fragile]
  \frametitle{Idle vs. Active}
  \centering
\begin{lstlisting}
sudo ./cpudist
Ctrl-C
  KERNEL 
   value  --------- Distribution --------- count    
   256 |                                     0        
   512 |                                     3        
  1024 |@@@@@@@@                             58       
  2048 |@@@@@@@@@@@@@                        93       
  4096 |@@@@@@@@@@@@@@@@                     120      
  8192 |@@                                   17       
 16384 |                                     1        
 32768 |@                                    4        
 65536 |                                     1        
131072 |                                     0        
\end{lstlisting}
\end{frame}

\begin{frame}
  \frametitle{A look inside cpudist}
  
\end{frame}

\begin{frame}[fragile]
  \frametitle{Changing Priorities}
\begin{lstlisting}
dtrace -n 'sched:::change-pri { printf("%s %d %d", execname, curlwpsinfo->pr_pri, arg2); }' | head
dtrace: description 'sched:::change-pri ' matched 1 probe
CPU     ID                    FUNCTION:NAME
  1  49443                      :change-pri csh 176 152
  1  49443                      :change-pri ls 176 120
\end{lstlisting}
\end{frame}

\begin{frame}
  \frametitle{A Multi-core World}
  \begin{itemize}
  \item All large systems are multi-core
  \item Scheduling on multi-core is difficult
  \item Some systems resort to static allocation
  \end{itemize}
\end{frame}

\begin{frame}[fragile]
  \frametitle{Are threads migrating?}
  \begin{itemize}
  \item Watching threads with \verb+cpuwalk.d+
  \end{itemize}
\end{frame}

\begin{frame}[fragile]
  \frametitle{Context Switching}
  \begin{itemize}
  \item Processes all believe they own the computer
  \item Context switching maintains this fiction
  \item Requires saving and restoring state
  \item Common measure of operating system performance
  \item \verb+cswstat.d+ measures overall context switching
  \end{itemize}
\end{frame}

\begin{frame}
  \frametitle{A look inside cswstat.d}
  
\end{frame}

\begin{frame}[fragile]
  \frametitle{Scheduler Lab Exercises}
  \begin{itemize}
  \item Write a one-liner to show processes waking up
  \item Extend wake up one-liner to include stack tracing
  \item Extend priority one-liner to include stack tracing
  \item Add periodic output to \verb+cpuwalk.d+
  \item Track context switching for a single process
  \end{itemize}
\end{frame}

\end{document}

%%% Local Variables:
%%% mode: latex
%%% TeX-master: t
%%% End:
