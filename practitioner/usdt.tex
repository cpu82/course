\documentclass[pdftex]{beamer}
\usetheme{metropolis}

\usepackage[english]{babel}
\usepackage[latin1]{inputenc}
\usepackage{times}
\usepackage[T1]{fontenc}
\usepackage{fancyvrb}
\usepackage{listings}
\begin{document}
\lstset{language=C, escapeinside={(*@}{@*)}, numbers=left,
  basicstyle=\tiny, showstringspaces=false, showspaces=false, showtabs=false}

\title{DTrace for Developers}
\subtitle{Tracepoints Outside the Kernel}
\author[shortname]{George V. Neville-Neil}

\begin{frame}
  \titlepage
\end{frame}

\begin{frame}
  \frametitle{Tracing Programs}
  \begin{itemize}
  \item Much of DTrace has been about the kernel
  \item There is a lot more code in user space
  \item We can trace that code too
  \end{itemize}
\end{frame}

\begin{frame}[fragile]
  \frametitle{The pid provider}
  \begin{itemize}
  \item Let's you look into a process
  \item The \verb+pid+ and \verb+target+ variables
  \item Functions work like \verb+fbt+ the kernel
  \end{itemize}
\end{frame}

\begin{frame}
  \frametitle{Finding probes in user space}
  \begin{itemize}
  \item Run the program with -c
  \item Attach to a running daemon with -p
  \end{itemize}
\end{frame}

\begin{frame}
  \frametitle{What probes exist in ls?}
  
\end{frame}

\begin{frame}[fragile]
  \frametitle{Whoops, what happened there?}
  \begin{itemize}
  \item DTrace continues to play it safe
  \item Protects the system against memory exhaustion
  \item Cut the probes in half by using \verb+:entry+
  \end{itemize}
\end{frame}

\begin{frame}[fragile]
  \frametitle{Dissecting a PID probe}
  \begin{description}
  \item[Provider] The PID
  \item[Module] Program or Library
  \item[Function] Function
  \item[Name] entry, return or a hex offset
  \end{description}
\begin{verbatim}
pid3722:ls:usage:entry  
pid3722:libc.so.7:__malloc:entry
\end{verbatim}
\end{frame}

\begin{frame}[fragile]
  \frametitle{Tracing an Instruction}
  \begin{itemize}
  \item We can create a probe point on an instruction
  \item Listing a function without a \verb+entry+ or \verb+return+
  \item Instructions are hexadecimal offsets
  \end{itemize}
\end{frame}

\begin{frame}[fragile]
  \frametitle{The instructions in malloc}
\begin{lstlisting}
dtrace -ln 'pid$target::__malloc:' -c ls
\end{lstlisting}
\end{frame}

\begin{frame}[fragile]
  \frametitle{Getting the return value}
  \begin{itemize}
  \item User space follows the same convention as kernel \verb+fbt+s
    \begin{itemize}
    \item Because they all follow the same ABI
    \end{itemize}
  \item \verb+arg0+ Return address
  \item \verb+arg1+ Return value
  \end{itemize}
\end{frame}

\begin{frame}[fragile]
  \frametitle{Tracing malloc()}
\begin{lstlisting}
dtrace -qn 'pid$target::malloc:return \
    { printf ("allocated 0x%x returned from offset 0x%x", arg1, arg0); }' -p 600
allocated 0x8040431c0 returned from offset 0x8e
\end{lstlisting}
\end{frame}

\begin{frame}[fragile]
  \frametitle{Does malloc ever complete?}
\end{frame}

\begin{frame}[fragile]
  \frametitle{Predicates and PIDs}
\begin{lstlisting}
dtrace -qn 'pid$target::malloc:return /arg0 != 0x8e/ 
   { printf ("allocated 0x%x returned from offset 0x%x\n", arg1, arg0); }' -c ls
\end{lstlisting}
\end{frame}

\begin{frame}[fragile]
  \frametitle{PID Provider Lab}
  \begin{itemize}
  \item How many probe points exist in \verb+ping+ ?
  \item What libraries are used by \verb+vi+?
  \item What does the \verb+usage()+ function return?
  \item Which function in \verb+ls+ calls \verb+malloc+ most frequently?
  \item Write a D script to track \verb+malloc()+ pointers.
  \end{itemize}
\end{frame}


\begin{frame}[fragile]
  \frametitle{The USDT Provider}
  \begin{itemize}
  \item User Space Dynamic Tracing
  \item Different from the \verb|pid| provider
  \item Add probes to your own programs
  \item Dynamic logging!
  \end{itemize}
\end{frame}

\begin{frame}[fragile]
  \frametitle{Adding Tracepoints to User Programs}
  \begin{itemize}
  \item Like \verb+SDT+ but different.
  \item More powerful in some ways
  \item Less powerful in others
  \end{itemize}
\end{frame}

\begin{frame}[fragile]
  \frametitle{Hello World, again}
\begin{lstlisting}
#include <stdio.h>
#include <unistd.h>

int main (int argc, char **argv) {
	int i;
	for (i = 0; i < 5; i++) {
		printf("Hello world\n");
		sleep(1);
	}
}  
\end{lstlisting}
  \begin{itemize}
  \item Thanks to Alan Hargreaves who wrote this for the DTrace book
  \end{itemize}
\end{frame}

\begin{frame}[fragile]
  \frametitle{Adding a single probe}
\begin{lstlisting}
#include <stdio.h>
#include <unistd.h>
#include <sys/sdt.h> /* <- new header file */

int main (int argc, char **argv) {
	int i;
	for (i = 0; i < 5; i++) {
		DTRACE_PROBE1(world, loop, i); /* <- probe point */
		printf("Hello world\n");
		sleep(1);
	}
}  
\end{lstlisting}
\end{frame}

\begin{frame}[fragile]
  \frametitle{Probe Description File}
\begin{lstlisting}
provider world {
        probe loop(int);
};

#pragma D attributes Evolving/Evolving/Common provider world provider
#pragma D attributes Private/Private/Common provider world module
#pragma D attributes Private/Private/Common provider world function
#pragma D attributes Evolving/Evolving/Common provider world name
#pragma D attributes Evolving/Evolving/Common provider world args
\end{lstlisting}
\end{frame}

\begin{frame}[fragile]
  \frametitle{Updated Build Process}
\begin{lstlisting}
cc -c hello.c
dtrace -G -s probes.d hello.o
cc -o hello -ldtrace probes.o hello.o  
\end{lstlisting}
\end{frame}

\begin{frame}
  \frametitle{Dissecting a USDT Probe}
  
\end{frame}

\begin{frame}
  \frametitle{Stability}
  \begin{description}
  \item[Name] Do we think the name will change?
  \item[Data] Are we committed to the data format?
  \item[Dependency] Is this probe OS or hardware dependent?
  \end{description}
\end{frame}

\begin{frame}
  \frametitle{Stability}
  \begin{description}
  \item [Internal] Part of DTrace
  \item [Private]  Vestige of Sun Microsystems
  \item [Obsolete] Will be removed in upcoming release, do not use.
  \item [External] Vestige of Sun Microsystems
  \item [Unstable] Can change at any time.
  \item [Evolving] Could change but becoming more stable.
  \item [Stable] Will not change within a major revision.
  \item [Standard] Defined by a standards body (POSIX, IETF etc.)
  \end{description}
\end{frame}


\begin{frame}
  \frametitle{Dependency Classes}
  \begin{description}
  \item[Unknown]
  \item[CPU] SPARC, Intel
  \item[Platform] FreeBSD, Illumos, MacOS
  \item[Group] Similar to ISA on FreeBSD
  \item[ISA] Instruction set architecture (amd64, arm32)
  \item[Common] Nearly all user space probes should use this.
  \end{description}
\end{frame}

\begin{frame}[fragile]
  \frametitle{Reviewing our probe file}
\begin{lstlisting}
provider world {
	probe loop(int);
};

#pragma D attributes Evolving/Evolving/Common provider world provider
#pragma D attributes Private/Private/Common provider world module
#pragma D attributes Private/Private/Common provider world function
#pragma D attributes Evolving/Evolving/Common provider world name
#pragma D attributes Evolving/Evolving/Common provider world args
\end{lstlisting}
\end{frame}

\begin{frame}[fragile]
  \frametitle{USDT Translator Lab}
  \begin{itemize}
  \item Install /usr/src onto your VM
  \item Add the following probes to \verb|/usr/src/bin/echo/echo.c|
    \begin{itemize}
    \item Probe the value of \verb|nflag|.
    \item Probe the value of \verb|len| in the argument loop.
    \item Probe the value of \verb|veclen| in the final loop.
    \end{itemize}
  \end{itemize}
\end{frame}

\end{document}

%%% Local Variables:
%%% mode: latex
%%% TeX-master: "developers"
%%% End:
