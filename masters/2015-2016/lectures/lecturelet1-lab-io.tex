% Included from both -slides and -handout versions.

\mode<presentation>
{
  \usetheme{default}
  \useoutertheme{infolines}
}

\usepackage[english]{babel}
\usepackage[latin1]{inputenc}
\usepackage{graphicx}
\usepackage{times}
\usepackage[T1]{fontenc}
\usepackage{fancyvrb}
\usepackage{hyperref}
\usepackage{listings}
\begin{document}
\lstset{language=C, escapeinside={(*@}{@*)}, numbers=left,
  basicstyle=\tiny, showspaces=false, showtabs=false}

\title{L41: Lab 1 - I/O}
\author{Dr Robert N. M. Watson}
\date{20 October 2015}

\begin{frame}
  \titlepage
\end{frame}

\section{Introduction}

\begin{frame}
  \frametitle{L41: Lab 1 - I/O}

  \begin{itemize}
    \item Introduce the BBB and DTrace
    \item Explore user-kernel interactions via syscalls and traps
    \item Learn a bit about POSIX I/O
    \item Measure the probe effect
  \end{itemize}
\end{frame}

\begin{frame}[fragile]
  \frametitle{The benchmark}

  \begin{scriptsize}
\begin{verbatim}
[guest@beaglebone ~/io]$ ./io-static 
io-static -c|-r|-w [-Bdqsv] [-b blocksize] [-t totalsize] path

Modes (pick one):
    -c              'create mode': create benchmark data file
    -r              'read mode': read() benchmark
    -w              'write mode': write() benchmark

Optional flags:
    -B              Run in bare mode: no preparatory activities
    -d              Set O_DIRECT flag to bypass buffer cache
    -q              Just run the benchmark, don't print stuff out
    -s              Call fsync() on the file descriptor when complete
    -v              Provide a verbose benchmark description
    -b blocksize    Specify a block size (default: 16384)
    -t totalsize    Specify total I/O size (default: 16777216)
\end{verbatim}
  \end{scriptsize}

  \begin{itemize}
    \item Simple, bespoke I/O benchmark: \texttt{read()} or \texttt{write()}
    \item Statically or dynamically linked
    \item Adjust buffer sizes, etc.
    \item Various output modes.
  \end{itemize}
\end{frame}

\begin{frame}
  \frametitle{The benchmark (2)}

  \begin{itemize}
    \item Three operational modes:
    \begin{description}
      \item[Create (\texttt{-c})] Create a new benchmark data file
      \item[Read (\texttt{-r})] Perform \texttt{read()}s against data file
      \item[Write (\texttt{-w})] Perform \texttt{write()}s against data file
    \end{description}
    \pause
    \item Adjust I/O parameters:
    \begin{description}
      \item[Block size (\texttt{-b})] Block size used for each I/O
      \item[Total size (\texttt{-t})] Total size across all I/Os
      \item[Direct (\texttt{-d})] Use direct I/O (bypass buffer cache)
      \item[Sync (\texttt{-s})] Perform \texttt{fsycnc()} after I/O loop
      \item[Bare (\texttt{-B})] Don't synchronise cache (etc) on start
	(whole-program testing)
    \end{description}
    \pause
    \item Output flags:
    \begin{description}
      \item[Quiet (\texttt{-q})] Suppress all output (whole-program tracing)
      \item[Verbose (\texttt{-v})] Verbose output (interactive testing)
    \end{description}
  \end{itemize}
\end{frame}

\begin{frame}[fragile]
  \frametitle{The benchmark (3)}

  \begin{scriptsize}
\begin{verbatim}
[guest@beaglebone ~/io]$ ./io-static -v -d -w /data/iofile 
Benchmark configuration:
  blocksize: 16384
  totalsize: 16777216
  blockcount: 1024
  operation: write
  path: /data/iofile
  time: 58.502746875
280.06 KBytes/sec
\end{verbatim}
  \end{scriptsize}

  \begin{itemize}
    \item Use verbose output
    \item Bypass the buffer cache
    \item Write to the previously created file \url{/data/iofile}
    \item Use default buffer size (16K) and total I/O size (16M)
  \end{itemize}
\end{frame}

\begin{frame}
  \frametitle{Exploratory questions}

  \begin{itemize}
    \item Baseline benchmark performance analysis:
    \begin{itemize}
      \item How do \texttt{read()} and \texttt{write()} performance compare?
      \item What is the performance impact of the buffer cache? \\
	Consider both \texttt{-d} and \texttt{-s}.
      \item What proportion of time is spent in userspace vs. the kernel?
      \item How many times are system calls invoked during the I/O loop?
      \item What is the role of traps in execution of the I/O loop?
      \item How does work performed in just the I/O loop compare with
	whole-program behaviour?
    \end{itemize}

    \pause
    \bigskip

    \item Probe effect and measurement decisions
    \begin{itemize}
      \item How does performance change if you insert system-call or trap
	probes in the I/O loop?
      \item What sources of variance may be affecting benchmark performance,
	and how can we measure them?
    \end{itemize}
  \end{itemize}
\end{frame}

\begin{frame}
  \frametitle{Experimental questions for the lab report}

  \begin{itemize}
    \item With respect to a configuration reading from a fixed-size file
      through the buffer cache:
    \begin{enumerate}
      \item How does changing the I/O buffer size affect I/O-loop performance?
      \item How does static vs. dynamic linking affect whole-program
	performance?
      \item At what file-size threshold does any performance difference
	between static and dynamic linking fall below 5\%? 1\%?
    \end{enumerate}

    \pause
    \bigskip

    \item Run the benchmark to gather initial measurements
    \item Explore through system-call/trap tracing and profiling
    \item Use various configurations (e.g., I/O on \url{/dev/zero}) to explore
      kernel code-path behaviour
  \end{itemize}
\end{frame}

\section{DTrace}

\begin{frame}[fragile]
  \frametitle{DTrace scripts}

  \begin{itemize}
    \item Human-facing C-like language
    \item One or more \{\textit{probe name}, \textit{predicate},
      \textit{action}\} tuples
    \item Expression limited to control side effects (e.g., no loops)
    \item Specified on command line or via a \texttt{.d} file
  \end{itemize}

  \pause
  \smallskip

  \begin{small}
\begin{verbatim}
fbt::malloc:entry /execname == "csh"/ { trace(arg0); }
\end{verbatim}
  \end{small}

  \smallskip

  \begin{description}
    \item[probe name] Identifies the probe(s) to instrument; wildcards
      allowed; identifies the \textit{provider} and a provider-specific
      \textit{probe name}
    \item[predicate] Filters cases where action will execute
    \item[action] Describes tracing operations
  \end{description}

\end{frame}

\begin{frame}
  \frametitle{Some kernel DTrace providers in FreeBSD}

  \begin{center}
  \begin{small}
  \begin{tabular}{ll}
    \hline
      \textbf{Provider} & \textbf{Description} \\
    \hline
      \texttt{callout\_execute} & Timer-driven callouts \\
      \texttt{dtmalloc} & Kernel \texttt{malloc()}/\texttt{free()} \\
      \textbf{\texttt{dtrace}} & DTrace script events (\texttt{BEGIN},
	\texttt{END}) \\
      \textbf{\texttt{fbt}} & Function Boundary Tracing \\
      \textbf{\texttt{io}} & Block I/O \\
      \texttt{ip}, \texttt{udp}, \texttt{tcp}, \texttt{sctp}  & TCP/IP \\
      \texttt{lockstat} & Locking \\
      \textbf{\texttt{proc}, \texttt{sched}} & Kernel process/scheduling \\
      \texttt{profile} & Profiling timers \\
      \textbf{\texttt{syscall}} & System call entry/return \\
      \textbf{\texttt{vfs}} & Virtual filesystem \\
    \hline
  \end{tabular}
  \end{small}
  \end{center}
\end{frame}

\begin{frame}
  \frametitle{Aggregations}

  \begin{center}
    \begin{small}
      \begin{tabular}{ll}
        \hline
        Aggregation & Description \\
        \hline
	\texttt{count()} & Number of times called \\
	\texttt{sum()} & Sum of arguments \\
	\texttt{avg()} & Average of arguments \\
	\texttt{min()} & Minimum of arguments \\
	\texttt{max()} & Maximum of arguments \\
	\texttt{stddev()} & Standard deviation ofnts \\
	\texttt{lquantize()} & Linear frequency distribution (histogram) \\
	\texttt{quantize()} & Log frequency distribution (histogram) \\
        \hline
      \end{tabular}
    \end{small}
  \end{center}
  \begin{itemize}
    \item Often we want summaries of events, not detailed traces
    \item DTrace allows early, efficient \textit{reduction} using aggregations
    \item Scalable multicore implementations (i.e., commutative)
    \item \texttt{@variable = function()}
    \item \texttt{printa()} to print
  \end{itemize}

\end{frame}

\begin{frame}[fragile]
  \frametitle{Counting kernel \texttt{read()} system calls}

  \begin{small}
\begin{verbatim}
[guest@beaglebone ~/io]$ ./io-static -q -r /data/iofile
\end{verbatim}
  \end{small}

\smallskip

  \begin{small}
\begin{verbatim}
root@beaglebone:/data/io # dtrace -n
  'syscall::read:entry
  /execname=="io-static"/
  {@reads = count(); }'
\end{verbatim}
  \end{small}

  \begin{description}
    \item[Probe] Trace the \texttt{read} system call
    \item[Predicate] Limit actions to processes executing \texttt{io-static}
    \item[Action] Count the number of probe fires
  \end{description}

  \pause

  \begin{small}
\begin{verbatim}
dtrace: description 'syscall::read:entry ' matched 1 probe
dtrace: buffer size lowered to 2m
dtrace: aggregation size lowered to 2m
^C

             1024
\end{verbatim}
  \end{small}
\end{frame}

\begin{frame}
  \frametitle{A few cautions}

  Copy key scripts and data files to/from your workstation

  \begin{itemize}
    \item The SD cards seem a bit fragile during poweroff -- make sure you
      shut down safely using the Lab Setup instructions
    \item We have spare imaged SD cards if you need them
    \item We may replace your SD cards for future labs
  \end{itemize}
\end{frame}

\begin{frame}
  \frametitle{A few other useful things}

  \begin{itemize}
    \item Work in pairs for the lab (reports must be written separately)
    \item Log in as \texttt{guest} on the console to set up an SSH key
    \item Otherwise, log in as \texttt{guest} via SSH
    \item Copy the lab bundle (see handout) to the BBB, and test
    \item In one terminal, \texttt{su} to \texttt{root} to run \texttt{dtrace}
    \item Then you will likely want multiple SSH sessions open
    \item The kernel source code is in \url{/usr/src/sys}
    \item Start with something simple -- e.g., DTrace \texttt{hello world}
    \item Do not hesitate to ask for help if you need a hand!
  \end{itemize}
\end{frame}

\end{document}
