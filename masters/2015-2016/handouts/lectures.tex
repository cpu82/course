\documentclass[a4paper,10pt]{article}
\usepackage{fullpage}
\usepackage{times}

\begin{document}
\title{L41: Lectures}
\author{Dr Robert N.M. Watson}
\date{Michaelmas Term 2015}
\maketitle

\noindent
L41 consists of six 1-hour lectures and five 2-hour labs.
This document provides a rough outline for the lectures, but also the brief
10--20-minute `lecturelets' at the start of each lab.

\subsection*{Lecture 1: Advanced Operating Systems: Tracing, Analysis, and
  Experimentation}

\begin{enumerate}
  \item Getting started
  \item What is an operating system?
  \item General-purpose operating systems
  \item Other kinds of operating systems
  \item What does an operating system do?
  \item Why study operating systems?
  \item Where is the OS research?
  \item What are the research questions?
  \item Teaching operating systems
  \item Aims of the module
  \item Prerequisites
  \item Module structure
  \item Rough module schedule and decomposition
  \item The platform
  \item Labs and lab reports
  \item Module texts - core material
  \item Module texts - supplemental material
\end{enumerate}

\subsection*{Lecture 2: Kernels and Tracing}

\begin{enumerate}
  \item Dynamic tracing with DTrace
  \item DTrace scripts
  \item D Intermediate Format (DIF)
  \item Some kernel DTrace providers in FreeBSD
  \item Tracing kernel malloc() calls
  \item Aggregations
  \item Profiling kernel malloc() calls by csh
  \item DTrace: implementation
  \item The `probe effect'
  \item Probe effect example: dd execution time
  \item Probe effect example 1: Memory allocation
  \item Probe effect example 2: Locking operations
  \item Probe effect example 2: Limiting to dd?
  \item Probe effect example 3: Locking stack traces
  \item The kernel: ``Just a C program''?
  \item The kernel: not just any C program
  \item Spelunking the kernel
  \item How work happens in the kernel
  \item Work processing and distribution
\end{enumerate}

\subsection*{Lab 1 - Lecturelet: I/O}

\begin{enumerate}
  \item The benchmark
  \item Exploratory questions
  \item Experimental questions for lab report
  \item DTrace scripts
  \item Some kernel DTrace providers in FreeBSD
  \item Aggregations
  \item Counting kernel read() system calls
  \item A few cautions
  \item A few other useful things
\end{enumerate}

\subsection*{Lecture 3: The Process Model (1)}

\begin{enumerate}
  \item The process model: 1970s foundations
  \item The process model: today
  \item The UNIX process life cycle
  \item Process model evolution
  \item Process address space: dd
  \item ELF binaries
  \item Virtual memory (quick but painful primer)
  \item Role of the run-time linker (rtld)
  \item Arguments and ELF auxiliary arguments
  \item Traps and system calls
\end{enumerate}

\subsection*{Lecture 4: The Process Model (2)}

\begin{enumerate}
  \item More about the process model
  \item System calls
  \item System-call synchrony
  \item System-call asynchrony
  \item System-call invocation from user to kernel
  \item The system-call table: syscalls.master
  \item Security and reliability
  \item System-call entry -- the guts: syscallenter
  \item getauid: return process audit ID
  \item System-call return -- the guts: syscallret
  \item System calls in practice: dd
  \item Traps in practice: dd
  \item So: back to virtual memory
  \item Last time: virtual memory (quick but painful primer)
  \item A (kernel) programmer model for virtual memory
  \item Mach VM in other operating systems
\end{enumerate}

\subsection*{Lab 2 - Lecturelet: IPC}

\begin{enumerate}
  \item Lab 2 - Kernel implications of IPC
  \item Recall: A (kernel) programmer model for VM
  \item The Mach VM fault handler (vm\_fault)
  \item The benchmark
  \item Exploratory questions - baseline performance
  \item Experimental questions for the lab report
  \item This lab session
\end{enumerate}

\subsection*{Lab 3 - Lecturelet: Micro-architectural implications of IPC}

\begin{enumerate}
  \item Lab 3 - Micro-architectural implications of IPC
  \item Hardware performance counters
  \item Sketch of ARM Cortex A8 memory hierarchy
  \item The benchmark -- now with PMC
  \item Example: Profile memory instructions
  \item Exploratory questions
  \item Experimental questions for the lab report
  \item This lab session
\end{enumerate}

\subsection*{Lecture 5: The Network Stack (1)}

\begin{enumerate}
  \item Introduction to the Network Stack
  \item A key OS function: networking
  \item The Berkeley Sockets API
  \item Early BSD network-stack design principles
  \item FreeBSD network-stack design principles
  \item Memory flow in hardware
  \item Memory flow in software
  \item The mbuf abstraction
  \item Local send/receive paths in the network stack
  \item Forwarding path in the network stack
  \item Work dispatch: input path
  \item Work dispatch: output path
  \item Work dispatch: TOE path
  \item netmap: a novel framework for fast packet I/O
  \item Network Stack Specialization for Performance
\end{enumerate}

\subsection*{Lecture 6: The Network Stack (2)}

\begin{enumerate}
  \item The Transmission Control Protocol (TCP)
  \item TCP goals and properties
  \item TCP congestion control and avoidance
  \item TCP time/sequence graphs
  \item BSD/FreeBSD TCP implementation evolution
  \item Lect 5: Local send/receive paths in the network stack
  \item Data structures - sockets, control blocks
  \item Denial of Service (DoS) - state minimisation
  \item TCP-connection lookup tables
  \item Lect 5: Work dispatch: input path
  \item An Evaluation of Network Stack Parallelization Strategies in Modern
    Operating Systems
  \item FreeBSD connection groups, RSS
  \item Performance: dispatch model and locking
  \item From architectural to micro-architectural optimisation
  \item Labs 4 + 5: TCP
\end{enumerate}

\subsection*{Lab 4 - Lecturelet: The TCP State Machine}

\begin{enumerate}
  \item Lab 4 - The TCP State Machine
  \item Lect 6: The Transmission Control Protocol (TCP)
  \item Lect 6: TCP goals and properties
  \item Loopback interface, IPFW, and DUMMYNET
  \item TCP in the IPC benchmark
  \item DTrace probes
  \item Exploratory questions
  \item Experimental questions for the lab report
  \item This lab session
\end{enumerate}

\subsection*{Lab 5 - Lecturelet: TCP latency and bandwidth}

\begin{enumerate}
  \item Lab 5 - TCP latency and bandwidth
  \item Lect 6: TCP goals and properties
  \item Lect 6: TCP congestion control and avoidance
  \item Lect 6: TCP time/sequence graphs
  \item Lect 6: Data structures - sockets, control blocks
  \item tcpcb sender-side data-structure fields
  \item Exploratory questions
  \item Experimental questions for the lab report
  \item This lab session
\end{enumerate}

\end{document}
