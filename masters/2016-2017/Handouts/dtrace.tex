\documentclass[a4paper,10pt]{article}
\usepackage{fullpage}
\usepackage{times}
\usepackage{upquote}

\begin{document}

\newcommand{\code}[1]{\texttt{\small #1}}

\title{L41: DTrace Quick Start}
\author{Dr Robert N.M. Watson}
\date{Michaelmas Term 2016}
\maketitle

DTrace is a dynamic tracing tool originally developed by Sun Microsystems (now
Oracle) available in Solaris, Mac OS X, FreeBSD, and as an optional add-on for
Linux.
A key design property of DTrace is that it is always available, possible
because it incurs no (serious) performance penalty when not being actively
used.
Simple tracing may be done using the command line; in practice, however, we
write short scripts in DTrace's D scripting language, which allow much more
complex behaviours such as conditional or speculative tracing, data
aggregation, and so on.
This document provides a short glossary of DTrace terms, information on the
\code{dtrace} command-line tool, a few sample DTrace scripts to get you
started, and a list of commonly used probes and built-in variables.

\section*{DTrace principles}

DTrace scripts consist of a series of statements identifying the
\textit{probes}, or instrumentation points, that should be enabled, an
optional \textit{predicate} that will further refine the set of behaviours to
trace, and an \textit{action} block that dictates what DTrace should do (e.g.,
trace) when a matching probe fires and is accepted by the predicate.
Probe names include implicit wildcards, making it easy to trace, for example,
all system calls, or all NFS client RPCs.
Predicates allow environmental context or retained script state to control
whether the action executes -- e.g., whether the process name is \code{sshd},
or if the same thread has previously performed some action during the current
system call.
Actions can record values in variables, generate trace output, and various
other activities that influence both future tracing and also, potentially,
system behaviour.
Your first DTrace script might read as follows:

\begin{small}
\begin{verbatim}
BEGIN { printf("Hello world"); exit(0); }
\end{verbatim}
\end{small}

This script matches the DTrace-internal \code{BEGIN} event, corresponding to
script start, and has two statements in its action block: one, to print the
string \code{Hello world}, and a second to terminate the script.
A more enlightening script might trace system-call arguments to the
\code{open()} system call by processes with the name \code{sshd}, using a
predicate to ensure that only events triggered by \code{sshd} will be traced:

\begin{small}
\begin{verbatim}
syscall::open:entry /execname == "sshd"/ { trace(copyinstr(arg0)); }
\end{verbatim}
\end{small}

\noindent
Notice that the system-call argument, \code{arg0}, is in fact a pointer to
user memory, rather than an in-kernel string.
The DTrace \code{copyinstr()} function allows the script to copy the string
into kernel memory so that its contents can be traced.

\section*{DTrace command}

The \code{dtrace} command-line tool provides a simple user interface to
DTrace.
It will generally be used in one of three forms: to list available probes, to
specify simple probes, predicates, and actions on the command line, or with a
script holding a set of probes, predicates, and actions originating in a file.
Except for very simple investigations, you will almost always want to use the
latter.

You will need to run \code{dtrace} as root; you can use \code{su} to
switch to the root user if you have logged in as another user via SSH.
Most filesystems in our teaching setup are mounted read-only in order to avoid
wear on the flash, as well as discourage storing files on partitions that we
might overwrite if issuing software updates during the module.
You are encouraged to store trace data, and any other information of value,
such as scripts you write, in \code{/data}, and to copy important data off
the board to your notebook or workstation for analysis.

\subsection*{Listing available probes}

To list all available probes:

\begin{small}
\begin{verbatim}
dtrace -l
\end{verbatim}
\end{small}

\noindent
To list all probes in the \code{syscall} provider:

\begin{small}
\begin{verbatim}
dtrace -l -P syscall
\end{verbatim}
\end{small}

\subsection*{Specifying probes, predicates, and actions on the command line}

The following examples use the \code{-n} argument, which specifies tracing by
probe name.
To trace return values (file descriptors) from the \code{open} system call:

\begin{small}
\begin{verbatim}
dtrace -n 'syscall::open:return /errno == 0/ { trace(execname); trace(arg0); }'
\end{verbatim}
\end{small}

\noindent
To track the aggregate distribution of memory sizes requested via the kernel
\code{malloc} function:

\begin{small}
\begin{verbatim}
dtrace -n 'fbt:kernel:malloc:entry { @foo = quantize(arg0); }'
\end{verbatim}
\end{small}

\noindent
Count stack traces sampled using the 99Hz kernel profiling timer:

\begin{small}
\begin{verbatim}
dtrace -n 'profile-99 { @foo[stack()] = count(); }'
\end{verbatim}
\end{small}

\noindent
Run a DTrace script named \code{foo.d}:

\begin{small}
\begin{verbatim}
dtrace -s foo.d
\end{verbatim}
\end{small}

\noindent
It will sometimes be useful to specify the \code{-q} flag, for \textit{quiet}
output: this requests that all normal DTrace output be suppressed, with only
explicit output from functions such as \code{printf()} being displayed.

\section*{DTrace scripts}

The \code{dtrace} command also accepts a \code{-s} argument allowing a
file to be specified as the source of the script to run.
Scripts consist of a series of probes, predicates, and actions.
Special probes, \code{BEGIN} and \code{END}, fire as the script starts up
and terminates, and the function \code{exit()} can be used to terminate a
script from an action.

\subsection*{Profiling callouts}

Here is a sample script you can use to profile time spent in kernel callouts
(timed asynchronous events).
At the start of the callout, a per-thread variable \code{cstart} records the
timestamp that execution begins.
When the callout ends, the \code{quantize} aggregation is used to insert the
difference between the current time and the start time into a histogram.

\begin{small}
\begin{verbatim}
callout_execute:::callout-start
{
        self->cstart = vtimestamp;
}

callout_execute:::callout-end
{
        @length = quantize(vtimestamp - self->cstart);
}
\end{verbatim}
\end{small}

\noindent
Exercise: What would happen if we used \code{timestamp} or
\code{walltimestamp} instead of \code{vtimestamp}?

\subsection*{Profiling callouts over a 1-second interval}

This script extends our previous script to keep track of time spent in
particular callouts, identified by their function name
(\code{arg0->c\_func}) over a one-second interval.
Note the use of an associative array indexed by function name, the
\code{sum} aggregation, and the use of a one-second timer tick to 
print the aggregation using \code{printa()} and \code{clear} to reset the
array for the next interval.

\begin{small}
\begin{verbatim}
#pragma D option quiet

callout_execute:::callout-start
{
        self->cstart = vtimestamp;
}

callout_execute:::callout-end
{

        @callouts[((struct callout *)arg0)->c_func] = sum(vtimestamp -
            self->cstart);
}

tick-1sec
{
        printa("%40a %10@d\n", @callouts);
        clear(@callouts);
        printf("\n");
}

BEGIN
{
        printf("%40s | %s\n", "function", "nanoseconds per second");
}
\end{verbatim}
\end{small}

\noindent
Exercise: How would we extend this script to not just add up all time spent by
a particular callout function, but instead record a histogram of the execution
times of each function over each second?

\subsection*{Collect stack traces to privilege check failures}

This script counts the number of unique stack traces leading to privilege
failures, detected using the \code{priv} provider.

\begin{small}
\begin{verbatim}
priv::priv_check:priv-err
{
        @traces[stack()] = count();
}
\end{verbatim}
\end{small}

\noindent
Exercise: How would we modify this script to instead keep a count of
privilege-check failures by system call?

\section*{DTrace providers}

The following DTrace providers are available on most FreeBSD systems; others
may be available when specific modules are loaded.
To list probes available from a provider, use \code{dtrace -l -P provider},
where \textit{provider} is the provider name.

\begin{description}
\item[callout\_execute] The kernel's callout (timer) subsystem; used to
  schedule asynchronous (and often recurring) events such as I/O timeouts and
  TCP retransmission.
  The two probes are \code{callout-start} and \code{callout-end}.

\item[dtmalloc] DTrace kernel memory-allocation provider, which exposes probes
  for allocation and freeing of various memory types.
  The per-type probe names are \code{malloc} and \code{free}.
  Note that this provider does not cover UMA (slab) allocations, only those by
  the general kernel \code{malloc}.

\item[dtrace] DTrace script events: \code{BEGIN}, \code{END}, and
  \code{ERROR}.

\item[fbt] Function Boundary Tracing (FBT): dynamically inserted probes in the
  prologues and epilogues of most kernel functions.

\item[io] Block I/O tracing; four probes: \code{start}, \code{done},
  \code{wait-start}, and \code{wait-done}.

\item[ip] Internet Protocol tracing; two probes \code{send} and
  \code{receive} allow instrumentation of IP-layer send and receive events.

\item[lockstat] Kernel lock profiling; probes for lock acquire, contention,
  and release events across several kernel lock types: spinlocks, mutexes,
  reader-writer locks, and shared-exclusive locks.

\item[mac, mac\_framework] MAC Framework tracing: probes placed around dynamic
  security-policy registration, object labelling, and access-control events.

\item[nfscl] Probes for the Network File System (NFS) client: NFSv3 and NFSv4
  RPC start/finish events, and also access and attribute cache events.

\item[priv] Kernel privilege checks.

\item[proc] Process events such as creation/destruction, \code{exec}, and
  signal delivery.

\item[profile] Timer-driven probes at various frequencies, either timed to
  align with system clock ticks (\code{tick}) or to avoid aliasing with
  clock ticks (\code{profile}).

\item[sched] Kernel scheduler tracing; events as thread enqueue/dequeue,
  priority changes, context switches to and from threads, preemption events,
  thread sleep/wakeup.

\item[sctp] Stream Control Transport Protocol (SCTP) events such as packet and
  congestion-control events.

\item[syscall] System Call tracing: \code{entry} and \code{return} probes
  for each system call across all supported ABIs.

\item[tcp] Transport Control Protocol (TCP) events such as connection
  creation, state change, and packet send/receive.

\item[udp] Unreliable Datagram Protocol (UDP) events: send and receive.

\item[vfs] Virtual File System (VFS) events: abstract filesystem events above
  the layer of the specific filesystem implementation including vnode
  operations (VOPs) on individual files, but also name lookup and caching.

\item[vm] Virtual Memory (VM): low-memory events.

\item[xbb] Xen Block Backend events associated with serving block-storage
  events from other Xen domains.
\end{description}

\section*{Favoured built-in variables}

DTrace provides a large number of built-in variables that can be referenced by
scripts; a detailed list can be found in the DTrace documentation (see
\textit{L41 Readings}).

\begin{description}
\item[arg0..arg9, args[]] Arguments to the DTrace probe: for FBT entry, the
  function's actual arguments; for FBT return, its return value as
  \code{arg0}; for system calls, likewise arguments and return value; for
  other probes, typically data-structure pointers such as a pointer to the
  pertinent \code{callout} structure for \code{callout\_execute} probes.
  Use \code{arg0}..\code{arg9} for untyped integer values, and
  \code{args[]} for typed structures and pointers that can be dereferenced.

\item[caller] The program counter where the probe fired.

\item[cpu] The CPU number that the probe fired on.

\item[cwd] The current thread's working directory.

\item[errno] The error value of the last system call executed by the thread.

\item[execname] The name of the executing binary.

\item[pid] The process ID of the current process.

\item[probefunc] The function name portion of the current probe's description.

\item[probemod] The module name portion of the current probe's description.

\item[probename] The name portion of the current probe's description.

\item[tid] The thread ID of the current thread.

\item[timestamp] The current time in nanoseconds, from an arbitrary point.

\item[vtimestamp] The current virtual time of the thread, in nanoseconds,
  from an arbitrary point.

\item[walltimestamp] The number of nanoseconds since Epoch.
\end{description}

\section*{DTrace/armv7 caveats}

DTrace on the armv7 architecture has a number of limitations that it is
important to be aware of when using it on the BeagleBone Black board:

\begin{itemize}
\item \texttt{dtrace -c} does not work on armv7.
\item Only the first four arguments to each DTrace probe are available.
\end{itemize}

\section*{DTrace glossary}

\begin{description}
\item[provider] DTrace \textit{providers} implement classes of events that can
  be instrumented using DTrace; for example, the \textit{Function Boundary
  Tracing} provider allows the prologues and epilogues of all kernel functions
  to be instrumented, and the \textit{System Call} provider similarly allows
  all system-call entry and return events to be instrumented.

\item[probe] A \textit{probe} is an event that can be instrumented and traced
  -- e.g., a particular function calling or returning, the profiler firing, a
  specific named system call entering.
  Probes have uniquely identifying names in the form
  \code{provider:module:function:name}.
  Fields in probe names may be omitted to match multiple probes.

\item[module] Some DTrace providers further include the name of a kernel
  module or component in the \textit{module} field, making it easier to narrow
  down the set of probes to a particular subsystem.
  For example, the \textit{Network File System} provider exposes the NFS
  version number as the module component of its probe names, and the
  \textit{System Call} provider includes the ABI name.

\item[function] DTrace probes likewise include a \textit{function name} that
  further identifies the probe that will fire; in the case of the
  \textit{Function Boundary Tracing} provider, the \textit{function} will be
  the name of the function instrumented, whereas the actual probe name will be
  \code{entry} or \code{return}.

\item[consumer] A DTrace \textit{consumer} is any program that interacts with
  DTrace to monitor system behaviour.
  For our purposes, this is the \code{dtrace} command-line tool, but others
  also exist, such as the \code{lockstat} tool.

\item[predicate] DTrace \textit{predicates} allow a script to inspect the
  run-time environment to make a determination as to whether an action should
  be executed as a result of a probe firing.
  This allows tracing of an event to be predicated on environmental factors
  such as what process is executing or prior events that have occurred in the
  same system call.

\item[action] DTrace script \textit{actions} are sequences of tracing
  operations to perform when a probe fires and is accepted by a predicate.
  This might include writing data to a trace buffer, but can also include
  updating variables or, for certain classes of scripts, changing the
  execution of the kernel.

\item[variables] DTrace \textit{variables} allow state to be recorded and
  acted on within scripts; variables can be \textit{global}, affecting all
  instances of execution predicates and actions, \textit{thread-local}
  (\code{self}), visible only to the current thread, and
  \textit{clause-local} (\code{this}), visible only to the current action.

  Thread-local variables are especially useful, as they can be used to record
  sequential operations by a thread that determine later tracing activities.
  For example, a per-thread variable might be used to store the name of the
  system call in execution so that later, when other operations are recorded,
  the system-call name is available to be included in a trace entry, or a
  decision to trace information might be conditioned on which system call is
  in flight.

\item[scalar variables] \textit{Scalar variables} are simple values
  associated with names specified by the programmer.
  Scalers can hold values of various types, including integers, pointers, and
  strings, but also instances of data structures originating in the program
  being traced.
  For example, \code{x = 5} or \code{self->syscall = probefunc}.

\item[associative arrays] \textit{Associative arrays} allow sets of keys and
  associated values to be stored in a single variable.
  For example, \code{self->functions[probefunc]++} or
  \code{functions[execname,probefunc]++}.

\item[built-in variables] DTrace includes a number of \textit{built-in
  variables} that hold information about the context for a probe, such as the
  current process ID, processor ID, executable name, user ID, etc.

\item[external variables] It is possible for a DTrace script to refer to
  kernel variables, which can be done by prefixing the variable name with the
  \code{`} character.
  For example, \code{`bootflags}.

\item[aggregations] \textit{Aggregations} are special global variables that
  efficiently collect and reduce large volumes in data in a scalable way.
  For example, the \code{@count} aggregating function allows counters of
  events to be efficiently implemented in the presence of multiple cores: each
  core maintains its own instance of the counter, avoiding locking when
  updating the counter on any particular core, and DTrace will combine those
  per-core values into a single global value before presenting it to the user.
  Other aggregating functions include \code{sum}, \code{avg},
  \code{min}, \code{max}, \code{lquantize}, and \code{quantize}.
  The latter two aggregations record distribution information, allowing
  histograms to be displayed by DTrace.

\item[speculation] DTrace scripts may include \textit{speculation}, which
  allows a set of operations to be performed conditional on some later event
  that will cause them to \textit{commit} (or \textit{abort}).
  For example, a script might speculatively collect argument data during a
  system call, but only commit those recorded arguments to the trace if the
  system call returns a specific error.

\item[functions] DTrace \textit{functions} perform a variety of activies,
  including appending to the trace (\code{trace()}), printing out values using
  format strings (\code{printf()}, or aggregations using \code{printa()}), or
  gathering stack traces (\code{stack()}).

\item[destructive scripts] Most DTrace scripts will simply inspect execution;
  however, it is also possible to modify kernel execution -- e.g., to change
  scheduling, or overwrite in-memory values.
  Such scripts are normally used in testing -- e.g., to trigger kernel edge
  cases that are otherwise hard to reach.
  This includes stopping the running process, sending signals, overwriting
  memory in user processes, or panicking the kernel to allow further debugging
  using a kernel debugger.
  For the purposes of this course, use of destructive scripts is not
  recommended, as incautious use can easily lead to system crashes or data
  corruption.
\end{description}

\section*{Further information}

See the module reading list for further information; the \code{dtrace} man
page, FreeBSD handbook, and World Wide Web are all useful reference material.
The DTrace book by Gregg and Mauro has a more tutorial-like structure, and is
strongly recommended.

\end{document}
