\documentclass[a4paper,10pt]{article}
\usepackage{fullpage}
\usepackage{times}

\begin{document}
\title{L41: Lab Reports}
\author{Dr Robert N.M. Watson}
\date{Michaelmas Term 2016}
\maketitle

The purpose of a \textit{laboratory report}, is to document an experiment, its
results, and its interpretation.
The exact contents and format of a lab report vary by discipline but the
principle is common across both the hard and social sciences, and will
generally include: the motivations and starting assumptions of the experiment
(to frame later design decisions); the goal of the experiment (i.e., the
hypothesis); the experimental setup itself; the results and their
interpretation (including sources of error); any conclusions; and a
comparison with related work in the field.
Computer science, and especially systems research, adopts many of these
elements, although not always overtly in the format of a lab report.
In L41, we use the genre of the lab report to help impose structure on our
experimental work, and as the primary means of assessment.

You will write three lab reports as part of the course: one `practice run'
derived from the I/O tracing exercises in the first submodule, and two
assessed reports (weighted at 50\% each) describing IPC and networking
experiments from the remaining submodules.
The first lab report is intended to help you develop and get feedback on your
report writing style so that assessment of later reports can focus on
experimental setup, data analysis, and conclusions.
As the lab reports are the primary form of assessment for the module, it is
important that you invest a significant amount of time in writing and refining
the report, paying attention to detail and presentation.
This is especially true for the `practice run', which is your opportunity to
correct errors and omissions in your approach without an impact on your final
mark.

\section*{Contents}

For the purposes of L41, we require that lab reports (moderately) faithfully
follow this structure:

\begin{description}
\item[Title page] Experiment name; author name; date.
  A 1-paragraph abstract will provide a succinct summary of the report
  including the nature of the experiment and conclusions that have been drawn.
  (1 page)
\item[Introduction] Frames the lab report as well as provides its context and
  motivations. (1--2 paragraphs)
\item[Experimental setup and methodology] An exploration of the goals,
  hypotheses, experimental setup (including platform details), procedure
  used in the experiment, and details of any steps taken to mitigate potential
  error or problems; a figure may be appropriate. (1--2 page including
  figures)
\item[Results and discussion]
  The results obtained, graphs illustrating those results, and an exploration
  and interpretation of the results including important artefacts, validity of
  the results, and conclusions they lead us to.
  Tables and figures required to explain the results should be present in all
  reports; in most cases, performance graphs will be expected, but in some
  cases, it may also be suitable to include state-machine diagrams.
  This is the body of the report. (3--4 pages including figures)
\item[Conclusion]
  A summation of the results and thoughts on potential future directions.
  (1--2 paragraphs)
\item[References]
  For our purposes, references to material that contextualises the work, as
  well as to pertinent readings we have done and a brief literature review;
  outside of this course, we would also expect a review of similar work by
  others, especially where results or methodology differ.
\item[Appendices]
  Additional material that supplements prior sections -- e.g., it might be
  desirable, in explaining material in the body, to reference content such as
  scripts to perform experiments, additional data tables, or more detailed
  illustrations of an experimental setup.
\end{description}

Lab reports will typically be 5-10 pages including figures but excluding
appendices.
Appendices should be included only where they improve understanding of the
body of the report -- simple DTrace scripts are not appropriate to include,
but should they be extended (for example) to mitigate a surprising form of
measurement error, it may be appropriate to include them in an appendix that
is referenced from discussion in the body.

\section*{Style and presentation}

Lab reports must be clearly written, spell checked, and formatted to make them
easy for the reader to follow.
Given length limitations, they will of necessity be high-level presentations
of our experiments, and cannot explore every detail.
Particular attention should be paid to graphs and tables that will present the
results: axes must be labelled, scales should be selected with care to avoid
misunderstanding, and if, for example, there are clear artefacts of interest,
then an additional graph may be appropriate to explore those in greater
detail.
All graphs must be described in the body of the text, and also have a
suitable (but brief) caption.
In general, it will be important to include error bars or other error
information, and explain when confidence intervals have been used.

LaTeX will be used, ideally using the \textbf{article} document class and 10pt
times font; use of the course template for lab reports is recommended, but not
required.
The precise graphing package is up to you; all graphs must be vector-based
rather than raster images, and must be prepared such that they are clear even
if printed in black and white.
It may be appropriate to use diagramming packages such as \textbf{tikz}, code
rendering via the \textbf{listings} package,  and additional tools such as
\textbf{matplotlib}, \textbf{R}, and \textbf{graphviz} to analyse and present
results.

Students are cautioned that many of these tools are complex and subtle, and
when used incautiously have a tendency to consume all available time.
If you run into difficulties, seek help from the course instructor or one of
the teaching assistants -- and when in doubt, avoid exciting-sounding features
in LaTeX!

\section*{Assessment}

\begin{description}
\item[$<60\%$] Below the pass mark: extremely poor (or incomplete)
  experimental procedure or writeup that might include an incoherent
  description of the work, poor experimental practice that leads to incorrect
  results, failure to discuss potential sources of error, and/or poor data
  analysis that draws incorrect conclusions despite clear evidence to the
  contrary.
  This marking range will also be used if there is insufficient originality.
\item[$>=60\%$, $<75\%$] Pass but below distinction: adequately performed
  experimental procedure and writeup, but with a few (but not many) of the
  following problems:
  (1) the experimental approach will have been roughly right, but failed to
  avoid potential sources of error, used inadequate runs to manage variance,
  or failed to pursue an important behaviour or effect;
  (2) the writeup will have drawn reasonable conclusions, but failed to make
  proper use of statistics, failed to explore sources of error, or failed to
  investigate a surprising effect or result; or
  (3) graphs will present useful results but are unclear or disagree with the
  experimental analysis.
\item[$>=75\%$] Pass with distinction; most or all of the following hold:
  a superior writing style and clarity; strong experimental procedure and
  error analysis, in which surprising results or artefacts are adequately
  illustrated via graphs and explained in the text; and strong or even new
  insights into performance are gained.
\end{description}

\section*{Collaboration}

While collaborating on experimentation itself is permitted (and even
encouraged), students must independently write up and submit lab reports.
You will need to employ your own discretion, but a reasonable approach might
have pairs of students collaborate to understand the target software, work
together to develop experimental setups and scripts, and jointly run initial
experiments to debug them together.
Pairs might then part ways to complete the experiments, analyse the data and
actual error, produce any graphs, and write up the results.
Data in appendices of reports may indeed turn out to be substantially similar
between lab partners as a result of collaboration on the setup and design
choices to scripts, but excessive similarity of graphs, text, and analysis
may be penalised as plagiarism.

\end{document}
